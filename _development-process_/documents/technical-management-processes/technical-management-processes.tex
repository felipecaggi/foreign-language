\documentclass[11pt, twoside, a4paper]{book}

    % chapters numbered without interruption (numbering through parts)
    \makeatletter\@addtoreset{chapter}{part}\makeatother
    
    \begin{document}
    
        \title{Technical management processes of repository "foreign-language"}
        \author{Felipe Alves Matos Caggi}
        \maketitle
        
        \tableofcontents
        \newpage
        
        \chapter{INTRODUCTION}
        	
        	This document specifies the development process of the "foreign-language" system, providing the project manager with the necessary information for the planning and management of the project lifecycle.

            
            \section{Full view of this document}
            
                This introduction provides the information needed to make good use of this document, explaining its objectives and the conventions that have been adopted in the text, as well as contains a list of references to other related documents. The other sections present the specification of the "foreign-language" system and are organized as described below.
                
                \begin{itemize}
                
                    \item Section 2 - Project Planning Process: provides an overview of the system, characterizing the scope of the project management and technical activities, identifying process outputs, tasks and deliverables, establishing schedules for task conduct, including achievement criteria, and required resources	 to	accomplish tasks.
                    
                    \item Section 3 - Process Assessment and Control Process: specifies the status of the project, technical and process performance, helping ensure that the performance is according to plans and schedules, within projected budgets, to satisfy technical objectives.
                                        
                    \item Section 4 - Risk Management Process: provides information about risks of the project, describing process is to identify, analyze, treat and	monitor	the	risks continually.

                    \item Section 5 - Configuration Management Process: specifies all system elements what need to configuration management over the life cycle. Also, present the consistency management between a product and its associated configuration definition.
                    
                \end{itemize}						
                
            \section{Conventions, terms and abbreviations}
                
                The interpretation correct this document exige the knowledge of some conventionals and specific terms, that are describes next.	                
                
                \subsection{Identification of commits}
                
                	\begin{center}
                		$[<Job Function ID Key>\_<Activity ID Key>\_<Commit Counter Per Job function>] <Description Commit>$
                	\end{center}
                
                \subsection{Identification of tags for baseline or releases}

					\begin{center}
						Baseline tag													

            			$[tgbl\_<Job Function ID Key>\_<Configuration Item Id Key For Management>\_<Tag Counter Per Item Of Configuration>]$
					\end{center}					                			

        			\begin{center}
		   				Release tag					        			
        			
            			$[tgrl\_<Job Function ID Key>\_<Configuration Item Id Key For Management>\_<Tag Counter Per Item Of Configuration>]	$	
					\end{center}					                
					
            \section{References}
                    
                    Documents related to the "foreign-language" system and / or mentioned in the following sections:
            
        \chapter{PROJECT PLANNING PROCESS}
        
        	\section{Plan project and technical management}
        	
        		\section{Project phases}
        		
					The development process adopted for this project has in its scope the following phases:
        			
        			\begin{enumerate}
        				
						\item Conception: is the first phase of the process, in which it seeks to raise the key requirements and understand the system comprehensively. The results of this phase are usually a requirement and risk document, a high-level use case listing, and a use case-based development timeline.
						\item Elaboration: this phase involves detailed system analysis, domain modeling and system design using the design patterns.
						\item Construction: includes a part of implementation and testing.
        				\item Transition: Upon ready, the system will be deployed replacing the current system of either manual or computerized.
        				
        			\end{enumerate}
        			
        			As subções 2.2.1 a 2.2.4 apresentam o detalhamento de atividades em cada fase do processo de desenvolvimento adotado.
	        		\subsection{Concepção}

						The conception phase presents in its scope the following activities:	
							
	        			\begin{enumerate}
	        				
	        				\item $[busmdl]$ Business modeling (System overview)
	        					
								In this activity, all possible information about the system must be obtained. The product of this step will be \emph{system overview document}.        				
	        						
					        \item $[reqgat]$ Requirements Gathering
					        
					        	It corresponds to searching for all the possible information about the functions that the system must execute and the reticulations on which the system must operate. The product of this step will be the \emph{documento de requisitos}, main component of the software design.
					        	
							\item $[reqaly]$ Requirements Analysis
							
								It serves to structure and detail the requirements so that they can be approached in the elaboration phase for the development of other elements such as use cases, classes, and interfaces.

							\item $[hlucse]$ High-level use cases
								
								In this activity it is necessary to identify the main use cases of the system. The use cases devel cover the main business activities linked to the system that will be developed.
								
	        			\end{enumerate}
	        			
    	    		\subsection{Elaboração}
    	    		
    	    			A fase de elaboração apresenta em seu escopo as seguintes atividades:
	        			
	        			\begin{enumerate}
	        				
							\item Expansion of use cases								
							\item Determination of events and system responses
							\item Construction or refinement of the conceptual model
							\item Elaboration of operations contracts and system queries
							
	        			\end{enumerate}
	        			
        			\subsection{Construção}
        			
        			\subsection{Transição}

        \chapter{PROCESS ASSESSMENT AND CONTROL PROCESS}
                    
        \chapter{RISK MANAGEMENT PROCESS}
        
        \chapter{CONFIGURATION MANAGEMENT PROCESS}
    
    \end{document}