\documentclass[11pt, twoside, a4paper]{book}

    % chapters numbered without interruption (numbering through parts)
    \makeatletter\@addtoreset{chapter}{part}\makeatother
    
    \begin{document}
    
        \title{Technical management processes of repository "foreign-language"}
        \author{Felipe Alves Matos Caggi}
        \maketitle
        
        \tableofcontents
        \newpage
        
        \chapter{INTRODUCTION}
        	
        	This document specifies the development process of the "foreign-language" system, providing the project manager with the necessary information for the planning and management of the project lifecycle.

            
            \section{Full view of this document}
            
                This introduction provides the information needed to make good use of this document, explaining its objectives and the conventions that have been adopted in the text, as well as contains a list of references to other related documents. The other sections present the specification of the "foreign-language" system and are organized as described below.
                
                \begin{itemize}
                
                    \item Section 2 - Project Planning Process: provides an overview of the system, characterizing the scope of the project management and technical activities, identifying process outputs, tasks and deliverables, establishing schedules for task conduct, including achievement criteria, and required resources	 to	accomplish tasks.
                    
                    \item Section 3 - Process Assessment and Control Process: specifies the status of the project, technical and process performance, helping ensure that the performance is according to plans and schedules, within projected budgets, to satisfy technical objectives.
                                        
                    \item Section 4 - Risk Management Process: provides information about risks of the project, describing process is to identify, analyze, treat and	monitor	the	risks continually.

                    \item Section 5 - Configuration Management Process: specifies all system elements what need to configuration management over the life cycle. Also, present the consistency management between a product and its associated configuration definition.
                    
                \end{itemize}						
                
            \section{Conventions, terms and abbreviations}
                
                The interpretation correct this document exige the knowledge of some conventionals and specific terms, that are describes next.	                
                
                \subsection{Identification of commits}
                
                	\begin{center}
                		$[<Job Function ID Key>\_<Activity ID Key>\_<Commit Counter Per Job function>] <Description Commit>$
                	\end{center}
                
                \subsection{Identification of tags for baseline or releases}

					\begin{center}
						Baseline tag													

            			$[tgbl\_<Job Function ID Key>\_<Configuration Item Id Key For Management>\_<Tag Counter Per Item Of Configuration>]$
					\end{center}					                			

        			\begin{center}
		   				Release tag					        			
        			
            			$[tgrl\_<Job Function ID Key>\_<Configuration Item Id Key For Management>\_<Tag Counter Per Item Of Configuration>]	$	
					\end{center}					                
					
            \section{References}
                    
                    Documents related to the "foreign-language" system and / or mentioned in the following sections:
            
        \chapter{PROJECT PLANNING PROCESS}

        \chapter{PROCESS ASSESSMENT AND CONTROL PROCESS}
            
        \chapter{DECISION MANAGEMENT PROCESS}
        
        \chapter{RISK MANAGEMENT PROCESS}
        
        \chapter{CONFIGURATION MANAGEMENT PROCESS}
                
        \chapter{INFORMATION MANAGEMENT PROCESS}
        
        \chapter{MEASUREMENT PROCESS}
        
        \chapter{QUALITY ASSURANCE PROCESS}
    
    \end{document}