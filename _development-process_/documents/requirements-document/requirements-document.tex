\documentclass[11pt, twoside, a4paper]{book}

% chapters numbered without interruption (numbering through parts)
\makeatletter\@addtoreset{chapter}{part}\makeatother

 \usepackage{color}
 
  \usepackage{array}
 
 \definecolor{Blue}{rgb}{0,0,0.9}
 
 \definecolor{Red}{rgb}{1,0,0}

\begin{document}

	\title{Repository requirements document "foreign-language"}
	\author{Felipe Alves Matos Caggi}
	\maketitle
	
	\tableofcontents
	\newpage
	
	\chapter{INTRODUCTION}
		
		This document specifies the "foreign language" system, providing developers with the necessary information for the design and implementation, as well as for testing and system approval.
		
		\section{Full view of this document}
		
			This introduction provides the information needed to make good use of this document, explaining its objectives and the conventions that have been adopted in the text, as well as contains a list of references to other related documents. The other sections present the specification of the "foreign language" system and are organized as described below.
			
			\begin{itemize}
				\item Section 2 - System Overview: provides an overview of the system, characterizing its scope and describing its users.
				\item Section 3 - Functional requirements (use cases): it specifies all the functional requirements of the system, describing the flows of events, priorities, actors, inputs and outputs of each use case to be implemented.
				\item Section 4 - Non-functional requirements: specifies all non-functional system requirements, broken down into usability, reliability, performance, security, distribution, adequacy, and hardware and software requirements.
				\item Section 5 - User Interface Description: displays drawings, figures, or sketches of system screens.
			\end{itemize}						
			
		\section{Conventions, terms and abbreviations}
			
			The interpretation correct this document exige the knowledge of some conventionals and specific terms, that are describes next.
			
			\subsection{Identification of requirements}
			
				By convention, the reference to the requirements is made through the name of the subsection where they are described, followed by the identifier of the requirement, according to the scheme below:

				\begin{center}
					[requirement subsection.identifier name]	
				\end{center}						
			

				For example, the requirement [Data Retrieval.RF016] is described in a subsection called "Data Retrieval", in a block identified by the number [RF016]. The nonfunctional requirement [Reliability.NF008] is described in the section on non-functional reliability requirements, in a block identified by [NF008].
				
			\subsection{Priority of requirements}
			
				In order to establish the priority of the requirements, the denominations "essential", "important" and "desirable" have been adopted.
				
				\begin{itemize}
					\item Essential is the requirement without which the system does not go into operation. Essential requirements are requirements that must be implemented certainly.
					\item Important is the requirement without which the system goes into operation, but in an unsatisfactory way. Important requirements must be implemented, but if they are not, the system can be deployed and used anyway.
					\item Desirable is the requirement that does not compromise the basic functionalities of the system, that is, the system can function satisfactorily without it. Desirable requirements are requirements that can be left for later versions of the system if there is not enough time to implement them in the version being specified.	
				\end{itemize}

		\section{References}
				
				Documents related to the "foreign-language" system and / or mentioned in the following sections:
		
	\chapter{SYSTEM OVERVIEW}

		The "foreign language" system is a desktop application that aims to identify patterns in the English language.
The importance of the application for the users is characterized in order to transmit knowledge about the most used tags, facilitating the language learning. In addition to creating a database that can be used in the future in other applications.		
		
		\section{Abragency and relational systems}
		
			The "foreign-language" system will receive text entries, and then perform the pattern search based initially on the number of word repetitions and later of the combination repeats. Each text insertion, the words, and possible combinations will be transferred to a database, where another section of the application is responsible for selecting the main sentences obtained and also responsible for returning this information to the user.
			
			It is important to understand that the application will not make available the database created by a user to the other users of the system, that is, each user will have their personal bank of words and combinations based on their inserted texts. This division is taken in order to simplify the initial development of the system, however, the system will be developed in order to facilitate future evolutions, such as database sharing.
			  
  			\textcolor{Red}{List and briefly describe the other systems with which this system should interact, explaining, in a general way, the roles of each and the means of communication between them.}
  			
		\section{User description}
			
			The application will be used by ordinary users\footnote{The system will be constructed so the user does not need to have specific knowledge present in the area of information technology.}. So , from the definition of users, it is identified that the main difficulties of the user are linked to the vocabulary, making it difficult to identify content words\footnote{Content words are the most important words in a sentence. They express the idea of what we mean.} and functions words\footnote{Function words are words that help us connect important information. They are the words that make the sentences grammatically correct.}, implying in this way, in the difficulty of understanding the full significate of a prhase.
			
	\chapter{FUNCTIONAL REQUIREMENTS}
		
		\section{Evident Functional Requirements}
		
			\begin{description}

				\item[RF001] Register texts from user inputs.
				\item[RF002] To present texts registered in the system.
				\item[RF003] To present words cataloged in each text.
				\item[RF004] To present all words registered in the system.
				\item[RF005] To present the count of distinct words in each text.
				\item[RF006] To present the count of words in each text registered for the first time in the system.
				\item[RF007] To present the count of all words distincts registered in the system. 
				\item[RF008] To present the count of repetitions of words in each text.
				\item[RF009] To present the count of repetitions of words of all texts registered in the system.
				\item[RF010] To present the indices of repetitions of the words.
				\item[RF011] To present the words most repeated in each text.
				\item[RF012] To present the words most repeated in all the texts.
				\item[RF013] To present combinations of the words cataloged in each text.
				\item[RF014] To present all combinations of the words registered in the system.
				\item[RF015] To present count of combinations of the distinct words in each text.
				\item[RF016] To present count of combinations of the words in each text registered for the first time in the system.
				\item[RF017] To present the count of all the combinations of the distinct words registered in the system.
				\item[RF018] To present the count of repetitions of the combinations of the words in each text.
				\item[RF019] To present the count of repetition of the combinations of the words in all text registered in the system.
				\item[RF020] To present the indices of repetitions of the word combinations.
				\item[RF021] To present the word combinations most repeated in each text.
				\item[RF022] To present the word combinations most repeated in all the texts.
				\item[RF023] Import and export the registered database.
				\item[RF024] CRUD users.			

			\end{description}
			
		\section{Hidden Functional Requirements}
			
			\begin{description}

				\item[RF025] Separate and catalog all the words of the texts registered in the system.
				\item[RF026] Calculate the indices of repetition of the words.
				\item[RF027] Separate and catalog all combinations of words from the texts registered in the system.
				\item[RF028] Calculate the indices of repetitions of the word combinations.

			\end{description}
		
	\chapter{NON-FUCTIONAL REQUIREMENTS}
	
	\chapter{USER INTERFACE DESCRIPTION}

	\chapter{USE CASE SPECIFICATION}

		\section{Use Cases}
				
		% This section present the relationship between the requirements.

			\begin{description}
				\item[UC001] Register texts
					
					The user can register new texts in the system. Thus, the system needs to store the user's text inputs.

				\item[UC002] View the registered texts
					
					The user can view all registered texts. Thus, the system must present all registered texts.

				\item[UC003] View information about each registered texts
					
					Each registered text has individual detailed information that the user can view, such as the number of words and repeats present in the text.
					
				\item[UC004] View statistics of repetition of words and word combinations
					
					The system can generate a report containing the generalization of information of each registered text.
					
				\item[UC005] Import and export the registered database
					
					The user can import and export information from registered texts.
					
				\item[UC006] CRUD users
					
					The user can access their records in the system on different machines. In this way, the user must be logged in to access the system functionality.
					
			\end{description}

		\section{Expansion of use cases}
			
			\subsection{[UC001]}
				\begin{tabular}{|>{\centering\arraybackslash}m{3cm} |>{\arraybackslash}m{9cm}|}												   \hline
					Use-Case Name 			& Register texts																							\\ \hline
					Short Description  		& The user can register new texts in the system. Thus, the system needs to store the user's text inputs.    \\ \hline	
					Preconditions  			& The user need stay logged in the system      																\\ \hline
					Subsequent Conditions	& The registered texts     																					\\ \hline
					Basic Flow  			& 	1. [IN] The user selects to register text
											
												2. [OUT] The system displays a form for the user to enter the text
											
												3. [IN] The user enters text
										
												4. [OUT] The system records the user input and reports the user about the success								\\ \hline
					Alternative Flows  		&       																									\\ \hline
				\end{tabular}
			
			\subsection{[UC002]}			
				\begin{tabular}{|>{\centering\arraybackslash}m{3cm} |>{\arraybackslash}m{9cm}|}										   \hline
					Use-Case Name 			& [UC002] View the registered texts																	\\ \hline
					Short Description  		& The user can view all registered texts. Thus, the system must present all registered texts.    	\\ \hline	
					Preconditions  			& The user need stay logged in the system      														\\ \hline
					Subsequent Conditions	& The display of all registered texts																\\ \hline
					Basic Flow  			& 	1. [IN] The user selects to view all registered texts														
											    
												2. [OUT] The system displays all registered texts														\\ \hline
				Alternative Flows  		&       																								\\ \hline
			\end{tabular}

			\subsection{[UC003]}			
				\begin{tabular}{|>{\centering\arraybackslash}m{3cm} |>{\arraybackslash}m{9cm}|}										   			   									   \hline
					Use-Case Name 			& [UC003] View information about each registered texts																								\\ \hline
					Short Description  		& Each registered text has individual detailed information that the user can view, such as the number of words and repeats present in the text.    	\\ \hline	
					Preconditions  			& The user need stay logged in the system      																										\\ \hline
					Subsequent Conditions	& The information about selected text disponibilized for user																						\\ \hline
					Basic Flow  			& 	1. [OUT] The system displays all registered texts	(UC002)													
											  
												2. [IN] The user selects text to view information														
											  
												3. [OUT] The system displays an individual report containing information about the selected text														\\ \hline
				Alternative Flows  		&       																																				\\ \hline
			\end{tabular}
			
			\subsection{[UC004]}			
				\begin{tabular}{|>{\centering\arraybackslash}m{3cm} |>{	\arraybackslash}m{9cm}|}										   			   \hline
					Use-Case Name 			& [UC004] View statistics of repetition of words and word combinations											\\ \hline
					Short Description  		& The system can generate a report containing the generalization of information of each registered text.    	\\ \hline	
					Preconditions  			& The user need stay logged in the system      																	\\ \hline
					Subsequent Conditions	& The display of the information about of all registered texts													\\ \hline
					Basic Flow  			& 	1. [OUT] The system displays all registered texts (UC002)   													
											  
												2. [IN] The user selects to view statistics of repetition of all registered texts											
											  
												3. [OUT] The system displays a report containing the generalization of the information of each registered text.	\\ \hline
				Alternative Flows  		&       																											\\ \hline
			\end{tabular}

\end{document}